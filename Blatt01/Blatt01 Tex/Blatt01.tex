\newcommand{\obenlinks}{Software Engineering}		% hier Name der Veranstaltung eintragen
\input{config.tex}
\usepackage{float}

\begin{document}
	\begin{center}
		\begin{tabular}{|rlp{4cm}rll|}
		\hline
		 \textbf{Übungsblatt:} & 1 &   & \textbf{1. Abgabepartner:} & Matthias Wolff & (458 766)  \\
		        & & & \textbf{2. Abgabepartner:} & Anton Mende & (461 328) \\
		        & & & \textbf{2. Abgabepartner:} & Anika Herbermann & (461 655) \\ \hline
		\end{tabular}
	\end{center}
\section*{Aufgabe 1}
\includegraphics[width=\textwidth,height=\textheight,keepaspectratio]{Aufgabe1.jpeg}
\section*{Aufgabe 2}
Die Auktion im folgenden Klassendiagramm funktioniert ähnlich wie eine Ebay-Auktion, nicht wie eine traditionelle, analoge Auktion. Es wird nur ein Gegenstand pro Auktion (Instanz der Gegenstand-Klasse) verkauft, nicht eine ganze Reihe von Gegenständen nacheinander, wie es in einem Auktionshaus üblich wäre (Die Aufgabenstellung schien hier nicht ganz eindeutig).\\
\includegraphics*[width=\textwidth,height=\textheight,keepaspectratio]{Aufgabe2.jpeg}
\section*{Aufgabe 3}
\includegraphics*[width=\textwidth,height=\textheight,keepaspectratio]{Aufgabe3.jpeg}
\end{document}