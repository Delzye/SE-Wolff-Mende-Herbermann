\newcommand{\obenlinks}{Software Engineering}		% hier Name der Veranstaltung eintragen
\input{config.tex}
\usepackage{float}

\begin{document}
	\begin{center}
		\begin{tabular}{|rlp{4cm}rll|}
		\hline
		 \textbf{Übungsblatt:} & 5 &   & \textbf{1. Abgabepartner:} & Matthias Wolff & (458 766)  \\
		        & & & \textbf{2. Abgabepartner:} & Anton Mende & (461 328) \\
		        & & & \textbf{2. Abgabepartner:} & Anika Herbermann & (461 655) \\ \hline
		\end{tabular}
	\end{center}
\section*{Aufgabe 15}
\section*{Aufgabe 16}
\begin{enumerate} [a)]
	\item 
	\begin{description}
		\item[Verarbeitung:] Nebenläufig, wenn der Föhn Sensoren für Temperatur etc. hat und die Föhnzeit misst.
		\item[Ein-/Mehrbenutzerfähigkeit:] Nur ein Benutzer, derjenige, der den Föhn nutzt, da das System des Föhns nur auf den Föhn selbst limitiert ist.
		\item[Schichten:] Kein Server, das ist zwar ein innovativer Föhn, aber kein Smart-Home Gerät, kein WLAN und keine Synchronisation mit der Cloud.
		\item[Plattformen:] Eine, die Hardware des Föhns.
		\item[Bausteine:] 
		\item[Hilfesystem:] Das System hat nichts zu erklären.
		\item[DBMS:] Keine DB. Wenn man sich die Föhnzeit merken will, um dem Nutzer Statistiken wie den Durchschnitt zu geben, kann man eine Datei nutzen (da es eine sehr simple Struktur ist).
		\item[Benutzerschnittstelle:] GUI, der Föhn nutzt einen Touchscreen
		\item[Dienstleistungen:] Keine weiteren Dienstleistungen
	\end{description}
	\item 
		\begin{description}
		\item[Verarbeitung:] Sequenziell: Alle Aktivitäten, Zutatenwahl, Synchronisation erfolgen nacheinander.
		\item[Ein-/Mehrbenutzerfähigkeit:] Die App hat immer nur einen Nutzer, nur der Server, der aber nicht Teil der App ist, muss mehrere Nutzer verarbeiten können.
		\item[Schichten:]
		\item[Plattformen:] iOS, Android. Also alle gängigen Smartphone-Betriebssysteme.
		\item[Bausteine:]
		\item[Hilfesystem:] Informationen zu Allergenen etc.
		\item[DBMS:] Wir nutzen ein relationales DBMS, um die Pizza-Daten zu synchronisieren.
		\item [Benutzerschnittstelle:] GUI der App.
		\item[Dienstleistungen:] Maps, um Die Pizzeria zu finden.
	\end{description}
	\item 
		\begin{description}
		\item[Verarbeitung:] Sequenziell, alle Bestandteile des Registrierens und Bestellens der Karte erfolgen nacheinander.
		\item[Ein-/Mehrbenutzerfähigkeit:] Mehrere Benutzer sollen sich gleichzeitig registrieren können und auch gleichzeitig Karten anfordern können.
		\item[Schichten:]
		\item[Plattformen:] Das System läuft auf einem Server, es muss also nur das Server-OS unterstützt werden.
		\item[Bausteine:]
		\item[Hilfesystem:] Informationen zur Verarbeitung der persönlichen Daten, Erklärung der Kartenfunktionen
		\item[DBMS:] Eine relationale Datenbank wird zum Speichern der Nutzerdaten genutzt.
		\item[Benutzerschnittstelle:] Website via HTML.
		\item[Dienstleistungen:] PayPal/Banksysteme zum Bezahlen, Verschlüsselung zum Transport der Kundendaten.
	\end{description}
	
\end{enumerate}
\end{document}