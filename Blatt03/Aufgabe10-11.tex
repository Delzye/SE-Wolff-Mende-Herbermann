\newcommand{\obenlinks}{Software Engineering}% hier Name der Veranstaltung eintragen

git 
\newcommand{\name}[1]{$\backslash$ #1 $\backslash$ \quad}		

\input{config.tex}

\begin{document}
	\begin{center}
		\begin{tabular}{|rlp{4cm}rll|}
		\hline
		 \textbf{Übungsblatt:} & 3 &   & \textbf{1. Abgabepartner:} & Matthias Wolff & (458 766)  \\
		        & & & \textbf{2. Abgabepartner:} & Anton Mende & (461 328) \\
		        & & & \textbf{2. Abgabepartner:} & Anika Herbermann & (461 655) \\ \hline
		\end{tabular}
	\end{center}
%Selbstdefinierter Befehl \name{<Name>} für Funktionen, Daten und Leistungen
\section*{Aufgabe 8)}
\subsection*{1 Zielbestimmung}
Die Software soll eine graphische Benutzeroberfläche zur Bestellung von Pizzen zur Verfügung stellen und die Bestellung zusammen mit den Kontaktdaten des Kunden an die Pizzeria übermitteln.
\subsection*{2 Anwendungsbereiche und Zielgruppen}
Die Software soll Kunden eine einfache Bestellung ihrer Pizzen ermöglichen.\\
Die Zielgruppe besteht aus Personen, die eine Pizza bestellen wollen
\subsection*{3 Produktfunktionen}
\name{LF10} Auswahl von Pizzen und ihrer Anzahl aus einer festen Pizzaliste\\
\name{LF20} Anmelden eines Benutzerkontos\\
\name{LF30} Erstellen / Registrieren eines Benutzerkontos\\
\name{LF40} Benachrichtigung an Kunden, falls Eingaben beim Anmelden / Registrieren ungültig\\
\name{LF41} Benachrichtigung an Kunden nach erfolgreicher Bestellung\\
\name{LF50} Übermittlung der Bestellung und Kontaktdaten des Bestellers an die Pizzeria
\subsection*{4 wichtigste zu speichernde Daten}
\name{LD10} Relevante Kundendaten sind zu speichern (Name, Adresse, Email, Telefonnummer,...)\\
\name{LD20} Die bestellbaren Pizzen sollen gespeichert werden\\
\name{LD30} Die erfolgten Bestellungen sind zu speichern\\
\subsection*{5 Leistungsanforderungen}
 \name{LL10} Es soll eine für eine Pizzeria übliche Anzahl an Kunden verwaltet werden\\
 \name{LL20} Die Software soll auf Benutzereingaben in unter 0,5 Sekunden reagieren\\
 \name{LL30} Die Übertragung der Bestellung an die Pizzeria soll in unter 10 Sekunden zuverlässig abgeschlossen sein
\subsection*{6 Qualitätsanforderungen}
\begin{tabular}{|c|c|}
\hline 
Merkmal & gewünschte Qualität \\ 
\hline 
Funktionalität & normal \\ 
Zuverlässigkeit & gut \\ 
Benutzbarkeit & sehr gut \\ 
Effizienz & normal \\ 
Änderbarkeit & normal \\ 
Portierbarkeit & gut \\ 
\hline 
\end{tabular}
\subsection*{7 Ergänzungen}
-
\subsection*{8 Glossar}
-

\section*{Aufgabe 10)}
\subsection*{a)}
context Vorfuehrung\\ \indent
\quad inv:\\ \indent
\qquad self.start <= self.ende\\ \indent
\qquad self.grundpreis > 0\\ \indent
\qquad self.name !=  \enquote{ }

\subsection*{b)}
context Bestellung \\ \indent
\quad inv: \\ \indent
\qquad Bestellung.allInstances -> select(best | best.vorfuehrung == self.vorfuehrung)\\ \indent \qquad -> forAll(best | best.reservierteSitze -> intersection(self.reservierteSitze) -> size == 0)
\newpage
\subsection*{c)}
context Bestellung::cancel()\\ \indent
\quad pre:\\ \indent
\qquad self.Zeitpunkt < vorfuehrung.start\\ \indent
\quad post: \\ \indent
\qquad not \quad Bestellung.allInstances -> exists(best | (best.nummer == self.nummer) and (best.vorfuerhung == self.vorfuehrung))

\subsection*{d)}
context Bestellung\\ \indent
\quad inv: \\ \indent
\qquad  (self.vorfuehrung.saal.nummer == 42) implies (self.vorfuehrung.ende < 11.11.2020)
\end{document}
