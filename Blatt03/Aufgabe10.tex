\newcommand{\obenlinks}{Software Engineering}		% hier Name der Veranstaltung eintragen
\input{config.tex}

\begin{document}
	\begin{center}
		\begin{tabular}{|rlp{4cm}rll|}
		\hline
		 \textbf{Übungsblatt:} & 3 &   & \textbf{1. Abgabepartner:} & Matthias Wolff & (458 766)  \\
		        & & & \textbf{2. Abgabepartner:} & Anton Mende & (461 328) \\
		        & & & \textbf{2. Abgabepartner:} & Anika Herbermann & (461 655) \\ \hline
		\end{tabular}
	\end{center}
\section*{Aufgabe 10}
\subsection*{a}
context Vorfuehrung\\ \indent
\quad inv:\\ \indent
\qquad self.start <= self.ende\\ \indent
\qquad self.grundpreis > 0\\ \indent
\qquad self.name !=  \enquote{ }

\subsection*{b}
context Bestellung \\ \indent
\quad inv: \\ \indent
\qquad Bestellung.allInstances -> select(best | best.vorfuehrung == self.vorfuehrung)\\ \indent \qquad -> forAll(best | best.reservierteSitze -> intersection(self.reservierteSitze) -> size == 0)
\subsection*{c}
context Bestellung::cancel()\\ \indent
\quad pre:\\ \indent
\qquad self.Zeitpunkt < vorfuehrung.start\\ \indent
\quad post: \\ \indent
\qquad not \quad Bestellung.allInstances -> exists(best | (best.nummer == self.nummer) and (best.vorfuerhung == self.vorfuehrung))

\subsection*{d}
context Bestellung\\ \indent
\quad inv: \\ \indent
\qquad  (self.vorfuehrung.saal.nummer == 42) implies (self.vorfuehrung.ende < 11.11.2020)
\end{document}
