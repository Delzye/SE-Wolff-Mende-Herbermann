\newcommand{\obenlinks}{Software Engineering}		% hier Name der Veranstaltung eintragen
\input{config.tex}
\usepackage{float}

\begin{document}
	\begin{center}
		\begin{tabular}{|rlp{4cm}rll|}
		\hline
		 \textbf{Übungsblatt:} & 3 &   & \textbf{1. Abgabepartner:} & Matthias Wolff & (458 766)  \\
		        & & & \textbf{2. Abgabepartner:} & Anton Mende & (461 328) \\
		        & & & \textbf{2. Abgabepartner:} & Anika Herbermann & (461 655) \\ \hline
		\end{tabular}
	\end{center}
\section*{Aufgabe 9}
\section*{Lastenheft: Pizza-Bestellsystem der Pizzaria Vesuv}
\section{Zielbestimmung}
Die zu entwickelnde Software soll die Bestellung von Pizzen, sowie die Anmeldung und/oder Registrierung von Kunden ermöglichen.
\section{Produkteinsatz}
Das Produkt dient der Annahme von Pizza-Bestellungen. Zielgruppe des Produkts sind die Kunden der Pizzeria Vesuv.
\section{Produktübersicht}
Umweltdiagramm
\begin{figure*}[htp]
	\centering
\includegraphics[keepaspectratio]{Umweltdiagramm.jpeg}
\caption {Umwelt des Produkts Pizza-Bestellsystem}
\end{figure*}
\section{Produktfunktionen}
\begin{enumerate}[/LF10/]
	\item Geschäftsprozess: Bestellung\\
	Akteur: Kunde\\
	Beschreibung: Ein Kunde wählt die für ihn relevanten Pizzen aus und gibt jeweils eine Anzahl an. Eine Anmeldung ist erforderlich.
	\item Geschäftsprozess: Anmeldung\\
	Akteur: Kunde\\
	Beschreibung: Ein Kunde meldet sich mit E-Mail und Passwort an.
	\item Geschäftsprozess: Registrierung\\
	Akteur: Kunde\\
	Beschreibung: Ein Kunde hinterlegt Name, Adresse, Telefonnummer, E-Mail-Adresse und ein Passwort.
\end{enumerate}
\section{Produktdaten}
\begin{enumerate}[/LD10/]
	\item Bestellungsdaten
	\item Kundendaten
\end{enumerate}
\section{Produktdaten}
\begin{enumerate}[/LL10/]
	\item Alle Reaktionszeiten auf Benutzereingaben müssen unter 0.5  Sekunden liegen.
\end{enumerate}
\section{Qualitätsanforderungen}
\begin{tabular}{l|cccc}
	Produktqualität&sehr gut&gut&normal&nicht relevant\\\hline
	Funktionalität&&&\texttimes&\\
	Zuverlässigkeit&&\texttimes&&\\
	Benutzbarkeit&\texttimes&&&\\
	Effizienz&&&&\texttimes\\
	Änderbarkeit&&&\texttimes&\\
	Übertragbarkeit&&&&\texttimes
\end{tabular}
\section{Ergänzungen}
keine
\section*{Glossar}
-
\section*{Aufgabe 10}
\begin{enumerate} [a)]
	\item \begin{enumerate} [1.]
		\item context Vorfuehrung \\
		inv: start <= end
		\item context Vorfuehrung\\
		inv: grundpreis > 0.0
		\item context Vorfuehrung \\
		inv: name <> “”
	\end{enumerate}
\item \phantom{a}\\
\begin{tabular}{ll}
	context Vorfuehrung&\\
	inv: Bestellung.allInstances ->&\\
	&select(b|b.Vorfuehrung = Vorfuehrung) ->\\
	&forAll(b1,b2 | b1<> b2 implies b1.reservierteSitze -> excludesAll(b2.reservierteSitze))\\
\end{tabular}
\item context Bestellung:: cancel()\\
pre: Zeitpunkt < Vorfuehrung.start
\item context Vorfuehrung \\
inv: saal.nummer = 42 implies ende < 11.11.2020
\end{enumerate}
\end{document}