\newcommand{\obenlinks}{Software Engineering}		% hier Name der Veranstaltung eintragen
\input{config.tex}
\usepackage{float}

\begin{document}
	\begin{center}
		\begin{tabular}{|rlp{4cm}rll|}
		\hline
		 \textbf{Übungsblatt:} & 3 &   & \textbf{1. Abgabepartner:} & Matthias Wolff & (458 766)  \\
		        & & & \textbf{2. Abgabepartner:} & Anton Mende & (461 328) \\
		        & & & \textbf{2. Abgabepartner:} & Anika Herbermann & (461 655) \\ \hline
		\end{tabular}
	\end{center}
\section*{Aufgabe 10}
\begin{enumerate} [a)]
	\item \begin{enumerate} [1.]
		\item context Vorfuehrung \\
		inv: start.Date <= end.Date
		\item context Vorfuehrung\\
		inv: grundpreis > 0.0
		\item context Vorfuehrung \\
		inv: name <> “”
	\end{enumerate}
\item \phantom{a}\\
\begin{tabular}{cl}
	context Vorfuehrung&\\
	inv: Bestellung.allInstances->&\\
	&select(b|b.Vorfuehrung.start = Vorfuehrung.start)->\\
	&forAll(b1,b2 | b1<> b2 implies b1.reservierteSitze → excludesAll(b2.reservierteSitze))\\
\end{tabular}
\item context Bestellung:: cancel()\\
pre: Zeitpunkt < Vorfuehrung.start
\item context Vorfuehrung \\
inv: saal.nummer = 42 implies ende < 11.11.2020
\end{enumerate}
\end{document}