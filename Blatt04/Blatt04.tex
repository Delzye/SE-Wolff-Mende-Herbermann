\newcommand{\obenlinks}{Software Engineering}		% hier Name der Veranstaltung eintragen
% Author: Phil Steinhorst, p.st@wwu.de
% https://github.com/phist91/latex-templates

\documentclass[%
	paper=a4,
	fontsize=10pt,
	ngerman
	]{scrartcl}

% Basics für Codierung und Sprache
% ===========================================================
	\usepackage{scrtime}
	\usepackage{etex}
	\usepackage{shellesc}					% Compiler-Option -shell-escape benutzen!
	\usepackage[final]{graphicx}			% Einbindung von Grafiken
	\usepackage[utf8]{inputenc}				% Dateien sind UTF8-codiert
	\usepackage{babel}						% deutsche Silbentrennung, etc.
	\usepackage[german=quotes]{csquotes}	% deutsche Anführungszeichen mit \enquote{...}
% ===========================================================

% Fonts und Typographie
% ===========================================================
	\usepackage{sourcecodepro}
	\usepackage[default]{sourcesanspro}
	\usepackage{nimbusmononarrow}
	
	\usepackage[babel=true,final,tracking=smallcaps]{microtype}
	\DisableLigatures{encoding = T1, family = tt* } % keine Ligaturen für Monospace-Fonts
	\usepackage{ellipsis}
% ===========================================================

% Farben
% ===========================================================
	\usepackage[usenames,x11names,final]{xcolor}
% ===========================================================

% Mathe-Pakete und -Einstellungen
% ===========================================================
	\usepackage{mathtools}					% Tools zum Setzen von Formeln
	\usepackage{amssymb}					% übliche Mathe-Symbole
	\usepackage[bigdelims]{newtxmath}		% moderne Mathe-Font
	\allowdisplaybreaks						% seitenübergreifende Rechnungen
	\usepackage{bm}							% math bold font
	\usepackage{wasysym}					% noch mehr Symbole
	\usepackage{forest}
	\usepackage{pifont}% http://ctan.org/pkg/pifont
% ===========================================================

% TikZ
% ===========================================================
	\usepackage{tikz}
	\usepackage{tikz-cd}					% kommutative Diagramme
	\usetikzlibrary{arrows.meta}			% mehr Pfeile!
	\usetikzlibrary{calc}					% TikZ kann rechnen
	\tikzset{>=Latex}						% Standard-Pfeilspitze
% ===========================================================

% Seitenlayout, Kopf-/Fußzeile
% ===========================================================
	\usepackage{scrpage2}
	\pagestyle{scrheadings}
	\usepackage[top=3cm, bottom=3cm, left=2.5cm, right=2cm]{geometry}
	\clearscrheadfoot 
	\setheadsepline{0.4pt}			 					% Linie in Kopfzeile
	\setfootsepline{0.4pt}								% Linie in Fußzeile
	\setkomafont{pagehead}{\bfseries}					% Schriftart Kopfzeile
	\setkomafont{pagefoot}{\normalfont\footnotesize}	% Schriftart Fußzeile 
	\cfoot{\thepage}									% Seitenzahl unten Mitte
	\lohead{\obenlinks}	% Titel oben links
	\raggedbottom							% Flattersatz
	\usepackage{setspace}					% erweiterte Abstandsoptionen
	\onehalfspacing							% Zeilenabstand 1.5-fach
	\setlength{\parindent}{0pt}				% Einrückung neuer Absätze
	\setlength{\parskip}{0.5\baselineskip}	% Abstand neuer Absätze
% ===========================================================

% Hyperref für Referenzen und Hyperlinks
% ===========================================================
	\usepackage[%
		hidelinks,
		pdfpagelabels,
		bookmarksopen=true,
		bookmarksnumbered=true,
		linkcolor=black,
		urlcolor=SkyBlue2,
		plainpages=false,
		pagebackref,
		citecolor=black,
		hypertexnames=true,
		pdfborderstyle={/S/U},
		linkbordercolor=SkyBlue2,
		colorlinks=false,
		backref=false]{hyperref}
	\hypersetup{final}
% ===========================================================

% Listen und Tabellen
% ===========================================================
	\usepackage{multicol}
	\usepackage[shortlabels]{enumitem}
	\setlist{itemsep=0pt}
	\setlist[enumerate]{font=\sffamily\bfseries}
	\setlist[itemize]{label=$\triangleright$}
	\usepackage{tabularx}
% ===========================================================

% listings
% ===========================================================
\usepackage{listingsutf8}
\lstset{
	belowcaptionskip=1\baselineskip,
	breaklines=true,
	showstringspaces=false,
	basicstyle=\ttfamily,
	keywordstyle=\bfseries\color{green!40!black},
	commentstyle=\itshape\color{purple!40!black},
	stringstyle=\color{orange},
	numbers=left,
	numberstyle=\footnotesize\ttfamily\color{gray},
	inputencoding=utf8/latin1,
	tabsize=4,
}

%%%%%%%%%%%%%%%%%%%%%%%%%%%%%%%%%%%%%%%%%%%%%%%%%%%%%%%%%%%
%%% Ab hier folgen nur noch vordefinierte Shortcuts %%%
%%%%%%%%%%%%%%%%%%%%%%%%%%%%%%%%%%%%%%%%%%%%%%%%%%%%%%%%%%%

\newcommand{\BB}{\mathbb{B}}
\newcommand{\CC}{\mathbb{C}}
\newcommand{\NN}{\mathbb{N}}
\newcommand{\QQ}{\mathbb{Q}}
\newcommand{\RR}{\mathbb{R}}
\newcommand{\ZZ}{\mathbb{Z}}
\newcommand{\oh}{\mathcal{O}}

\newcommand{\ol}[1]{\overline{#1}}
\newcommand{\wt}[1]{\widetilde{#1}}
\newcommand{\wh}[1]{\widehat{#1}}

\DeclareMathOperator{\id}{id} 				% Identität
\DeclareMathOperator{\pot}{\mathcal{P}}		% Potenzmenge

% Klammerungen und ähnliches
\DeclarePairedDelimiter{\absolut}{\lvert}{\rvert}		% Betrag
\DeclarePairedDelimiter{\ceiling}{\lceil}{\rceil}		% aufrunden
\DeclarePairedDelimiter{\Floor}{\lfloor}{\rfloor}		% aufrunden
\DeclarePairedDelimiter{\Norm}{\lVert}{\rVert}			% Norm
\DeclarePairedDelimiter{\sprod}{\langle}{\rangle}		% spitze Klammern
\DeclarePairedDelimiter{\enbrace}{(}{)}					% runde Klammern
\DeclarePairedDelimiter{\benbrace}{\lbrack}{\rbrack}	% eckige Klammern
\DeclarePairedDelimiter{\penbrace}{\{}{\}}				% geschweifte Klammern
\newcommand{\Underbrace}[2]{{\underbrace{#1}_{#2}}} 	% bessere Unterklammerungen
% Kurzschreibweisen für Faule und Code-Vervollständigung
\newcommand{\abs}[1]{\absolut*{#1}}
\newcommand{\ceil}[1]{\ceiling*{#1}}
\newcommand{\flo}[1]{\Floor*{#1}}
\newcommand{\no}[1]{\Norm*{#1}}
\newcommand{\sk}[1]{\sprod*{#1}}
\newcommand{\enb}[1]{\enbrace*{#1}}
\newcommand{\penb}[1]{\penbrace*{#1}}
\newcommand{\benb}[1]{\benbrace*{#1}}
\newcommand{\stack}[2]{\makebox[1cm][c]{$\stackrel{#1}{#2}$}}
\usepackage{float}

\begin{document}
	\begin{center}
		\begin{tabular}{|rlp{4cm}rll|}
		\hline
		 \textbf{Übungsblatt:} & 4 &   & \textbf{1. Abgabepartner:} & Matthias Wolff & (458 766)  \\
		        & & & \textbf{2. Abgabepartner:} & Anton Mende & (461 328) \\
		        & & & \textbf{2. Abgabepartner:} & Anika Herbermann & (461 655) \\ \hline
		\end{tabular}
	\end{center}
\section*{Aufgabe 12}
\begin{enumerate} [/LF10/)]
	\item Als Vorsitzender möchte ich mir übergebene oder geschickte schriftliche Beitrittserklärungen bearbeiten können
	\item  Als Vorsitzender möchte ich mir übergebene oder geschickte schriftliche Austrittserklärungen bearbeiten können
	\item Als Kassierer möchte ich einen Betrag von einem Konto auf ein anderes buchen können
	\item Als Vorstandsmitglied? möchte ich eine Liste allen Mitglieder mit negativen Kontoständen ausgeben lassen können
	\item Als Kassierer möchte ich einem Mitglied mehrere Beitragssätze zuordnen können
	\item Als Kassierer möchte ich einsehen können, seit wann einem Mitglied ein bestimmter Beitragssatz zugeordnet wurde
\end{enumerate}
\section*{Aufgabe 13}
Alle nicht explizit benannten Stellen und Kantenkapazitäten sind =1. Die Anfangsbelegung ist hier bereits dargestellt.
\begin{tikzpicture}[node distance=1.5cm,>=stealth',bend angle=45,auto]
\tikzstyle{place}=[circle, draw=black!85, minimum size =6mm]

\node [transition] (t1) [label=above:Lager 1]{} ;
\node [transition] (t2) [below of=t1,label=above:Lager 2]{} ;
\node [transition] (t3) [below of=t2, label=above:Lager 3] {};
\node [place,tokens=1] (i1) [left of=t1,label=above: Tomatensoße, label=below:\text{k=100}, node distance =1.7cm] {}
edge [pre] (t1);
\node [place,tokens=1] (i2) [left of=t2,label=above: Mozarella, label=below:\text{k=100}, node distance =1.7cm] {}
edge [pre] (t2);
\node [place,tokens=1] (i3) [left of=t3,label=above: Rohling, label=below:\text{k=100}, node distance =1.7cm] {}
edge [pre] (t3);

\node [transition] (p1) [below left of=i3, label=below:belegt]{}
edge [pre,bend left] (i1) 
edge [pre,bend left] (i2)
edge [pre,bend left] (i3);
\node [place] (pp) [below left of=p1, node distance=3cm,label=right: \text{k=2, Backprozess}] {}
edge[pre, bend left=27] (p1)
edge[post, bend left] (p1);

\node [transition]  (b1) [left of=pp,label=above: betritt] {}
edge [post] node{2}(pp);
\node [place, tokens=2] (bp1) [below left of=pp,node distance=4cm, label=right: \text{k=2, Mitarbeiter}] {}
edge [post]node{2} (b1);
\node [transition] (b4) [left of=b1, node distance=4cm, label=above: Bestellen] {}
edge [post, bend right =10] (bp1)
edge [pre, bend left=10] (bp1);
\node [place] (bp6) [below of=b4,label=left:bestellt] {}
edge [pre] (b4);
\node [place] (bp5) [left of=b4,label=above: Kunde] {}
edge [post] (b4);
\node [transition] (b6) [left  of=bp5,label=left: Kommt] {}
 edge [post] (bp5);

\node [place] (pp0) [below right of=p1, node distance=3cm,label=right: \text{k=4, belegte Pizzen}] {}
edge [pre, bend right] (p1);
\node [transition] (p2) [below right of=pp, node distance=3cm, label=above:legt in Ofen]{}
edge [pre, bend right] node {4}(pp0) 
edge[pre, bend left=27] node[swap]{2}(pp)
edge[post, bend left] (pp)
edge [post, bend right=5] (bp1);
\node [place] (pp1) [below of=p2,label=right:\text{k=4, Ofen}] {}
edge [pre] node{4} (p2);
\node [transition] (p3) [below of=pp1, label=right: backt]{}
edge [pre] node{4}(pp1)
edge [pre, bend left=28](pp) 
edge [post, bend left=15] (bp1) ;
\node [place] (pp2) [below of=p3,label=right: \text{k=8, Beheizte Ablage}] {}
edge [pre] node{4} (p3);
\node [transition] (p4) [left of=pp2,label=below: legt auf Arbeitsfläche]{}
edge [pre](pp2)
edge [pre,bend left=5] (bp1);
\node [place] (pp3) [left of=p4,node distance =3cm, label=below:\text{Arbeitsfläche}] {}
edge [pre] (p4);
\node [transition] (p5) [above left of=pp3,node distance=2cm, label=left: belegt]{}
edge [pre] (pp3)
edge [post] (bp1);

\node [place] (bp2) [above left of=p5,label=above right:\text{Theke}] {}
edge [pre] (p5);
\node [transition] (b2) [above left of=bp2, label=left: bezahlt] {}
edge [pre, bend left =10] (bp1)
edge [post, bend right =10] (bp1)
edge [pre] (bp6)
edge [pre] (bp2);
\node [place] (bp4) [below left of=b2,node distance =1.7cm,label=below: Kunde fertig] {}
edge [pre] (b2);
\node [place] (bp3) [below right of=bp4,label=below:\text{k=2000, Kasse}] {}
edge [pre] node{8} (b2);
\node [transition] (b3) [left of=bp4, node distance=2cm, label=below: geht] {}
edge [pre] (bp4);

\node [place,tokens=1] (pp4) [below left of=p5, node distance=3cm,label=above: Schinken,label=below:\text{k=100}] {}
edge [post, bend right] (p5);
\node [place,tokens=1] (pp5) [below of=pp4, label=above: Rucola,label=below:\text{k=100}] {}
edge [post, bend right] (p5);
\node [transition] (p6) [left of=pp4,label=above: Lager 4] {}
edge [post] (pp4);
\node [transition] (p7) [left of=pp5,label=above: Lager 5] {}
edge [post] (pp5);
\end{tikzpicture}
\section*{Aufgabe 14}
\begin{tikzpicture}[->,>=stealth',shorten >=1pt,auto,node distance=2.8cm,
semithick]
\node  [initial,state] (11000) {(1,1,0,0,0)};
\node[state]  (10110) [below of=11000] {(1,0,1,1,0)};
\node [state] (01110) [left of=10110]{(0,1,1,1,0)};
\node [state] (00011) [left of=01110]{(0,0,0,1,1)};
\node [state] (01010) [left of=00011] {(0,1,0,1,0)};
\node [state] (10100) [below of=10110] {(1,0,1,0,0)};
\node [state] (01100) [below of=10100] {(0,1,1,0,0)};
\node [state] (00001) [below of=01100] {(0,0,0,0,1)};
\node [state] (01000)[below of=00001]  {(0,1,0,0,0)};
\node [state] (00110)[below of=01000]  {(0,0,1,1,0)};
\node [state] (00100)[below of=00110]  {(0,0,1,0,0)};

\path(11000) edge node{$t_1$} (10110)
(10110) edge node{$t_0$} (01110)
			 edge node{$t_3$} (10100)
(01110) edge node{$t_2$} (00011)
			 edge node{$t_3$} (01100)
(00011) edge node{$t_3$} (00001)
			edge node{$t_0$} (01010)
(01010) edge node{$t_3$} (01000)
(10100) edge node{$t_0$} (01100)
(01100) edge node{$t_2$} (00001)
(00001) edge node{$t_4$} (01000)
(01000) edge node{$t_1$} (00110)
(00110) edge node{$t_3$} (00100);

\end{tikzpicture}
\end{document}

Das B/E-Netz ist deadlockfrei, aber nicht lebendig, da für eine vom Anfangszustand aus erreichbare Markierung (z.B. (0,0,1,0,1)) keine Markierung erreichbar ist, bei der $t_0$ aktiviert wird ($\rightarrow t_0$ ist nicht lebendig).\\
\begin{tikzpicture}[->,>=stealth',shorten >=1pt,auto,node distance=2.8cm,
semithick]
\node[initial,state] (A)                    {(1,1,0,0,0)};
\node[state]         (B) [below of=A] {(1,0,0,0,1)};
\node[state]         (D) [right of=A] {(1,0,1,1,0)};
\node[state]         (C) [above of=D] {(1,0,1,0,0)};
\node[state]         (E) [right of=D]       {(0,1,1,1,0)};
\node[state]         (F) [above of=E]       {(0,1,1,0,0)};
\node[state]         (G) [above of=F]       {(0,0,1,0,1)};
\node[state]         (K) [right of=E]       {(0,0,1,1,1)};
\node[state]         (H) [above of=K]       {(0,1,0,0,1)};
\node[state]         (I) [below of=E]       {(0,1,0,1,1)};
\node[state]         (M) [below right of=A]       {(1,1,0,1,0)};
\node[state]         (L) [below of=M]       {(1,0,0,1,1)};

\draw (B) .. controls  (1,-10) and (13.5,-1) .. node {$t_0$}(H);
\draw (C) .. controls (-3.5,3) and (-3.5,-2).. node [swap]{$t_0$} (B);
\path	(A) edge		[bend left]		node {$t_2$}	(B)
edge				node {$t_1$}	(D)
(B)	edge[bend left]	node {$t_4$}	(A)
(C) edge node {$t_0$} (F) 
(D) edge node {$t_3$} (C)
edge [bend left]node {$t_2$} (L)
edge node {$t_0$} (E)
(E) edge node {$t_3$} (F)
edge[bend right] node {$t_2$} (K)
edge node {$t_2$} (I)
(F) edge [bend left] node {$t_2$} (G)
edge node {$t_2$} (H)
(G) edge [bend left] node {$t_4$} (F)
(H)  edge [bend right] node {$t_2$} (K)
(I) edge [bend right =75]node {$t_2$} (H)
(K) edge [bend right] node {$t_3$} (H)
edge [bend right] node {$t_4$} (E)
(L) edge node [swap]{$t_0$} (I) 
edge node {$t_3$} (B) 
edge [bend right] node {$t_4$} (M)
(M) edge [bend right]node {$t_2$} (L) 
edge node [swap] {$t_3$} (A);
\end{tikzpicture}