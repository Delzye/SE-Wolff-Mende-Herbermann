%!TEX TS-program = pdflatex
%!TEX TS-options = -shell-escape
% Author: Phil Steinhorst, p.st@wwu.de
% https://github.com/phist91/latex-templates

\newcommand{\obenlinks}{Software Engineering}		% hier Name der Veranstaltung eintragen
\input{config.tex}	% Präambel (ohne die geht nichts!)
\usepackage{float}

\begin{document}
	\begin{center}
		\begin{tabular}{|rlp{4cm}rll|}
		\hline
		 \textbf{Übungsblatt:} & 6 &   & \textbf{1. Abgabepartner:} & Anton Mende & (461 328)  \\
		      &  &  & \textbf{2. Abgabepartner:} & Anika Herbermann & (461 655) \\ 
		      &  &  &  \textbf{3. Abgabepartner:} & Matthias Wolff & (458 766) \\
		      \hline
		\end{tabular}
	\end{center}
	\section*{Aufgabe 18 a+b)}
		Tabelle zur Klasse Kunde: \\
		\begin{tabular}{|lll|}
			\hline
			\underline{KundenId} & Vorname & Nachname \\
			\hline
			001 & Peter & Strautmann \\
			007 & James & Bond \\
			\hline
		\end{tabular}
		\\ \\
		Tabelle zur Klasse Bestellung: \\
		\begin{tabular}{|llll|}
			\hline
			\underline{BuchungsId} & Buchungszeitpunkt & KundenId & PizzaId\\
			\hline
			021 & 25:01:2016 & 007 & 042\\
			003 & 12:02:2018 & 001 & 036\\
			\hline
		\end{tabular}
		\\ \\
		Tabelle zur Klasse Pizza: \\
		\begin{tabular}{|lll|}
			\hline
			\underline{PizzaId} & Name & Preis\\
			\hline
			036 & Schinkenpizza & 3,50€ \\
			042 & Pizza Hawaii & 4,00€ \\
			\hline
		\end{tabular}
	\section*{Aufgabe 20}
		\subsection*{b)}
			\textbf{[rounds, rounds = 0, rounds++]} \\
			\textbf{[rounds, rounds ++, rounds ++]} \\
			\textbf{[rounds, rounds = 0, return rounds]} \\
			\textbf{[energy, energy = energy - 2, energy > 0]} \\
			\textbf{[energy, energy = energy - 1, energy > 0]}
		\subsection*{c)}
			\begin{enumerate}
				\item Testfall: energy = 0
				\item Testfall: energy = 3
			\end{enumerate}
\end{document}
