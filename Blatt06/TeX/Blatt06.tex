\newcommand{\obenlinks}{Software Engineering}		% hier Name der Veranstaltung eintragen
\input{config.tex}
\usepackage{float}

\begin{document}
	\begin{center}
		\begin{tabular}{|rlp{4cm}rll|}
		\hline
		 \textbf{Übungsblatt:} & 6 &   & \textbf{1. Abgabepartner:} & Matthias Wolff & (458 766)  \\
		        & & & \textbf{2. Abgabepartner:} & Anton Mende & (461 328) \\
		        & & & \textbf{3. Abgabepartner:} & Anika Herbermann & (461 655) \\ \hline
		\end{tabular}
	\end{center}
\definecolor{Gray}{gray}{0.85}
\definecolor{mymauve}{rgb}{0.58,0,0.82}
\definecolor{javared}{rgb}{0.6,0,0} % for strings
\definecolor{javagreen}{rgb}{0.25,0.5,0.35} % comments
\definecolor{javapurple}{rgb}{0.5,0,0.35} % keywords
\definecolor{javadocblue}{rgb}{0.25,0.35,0.75} % javadoc
\definecolor{pblue}{rgb}{0.13,0.13,1}
\definecolor{pgreen}{rgb}{0,0.5,0}
\definecolor{pred}{rgb}{0.9,0,0}
\definecolor{pgrey}{rgb}{0.46,0.45,0.48}


\lstset{language=Java,
	frame =single,
	basicstyle=\ttfamily,
	commentstyle=\color{pgreen},
	keywordstyle=\color{pblue},
	stringstyle=\color{pred},
	morecomment=[s][\color{javadocblue}]{/**}{*/},
	numbers=left,
	numberstyle=\tiny\color{black},
	stepnumber=3,
	numbersep=10pt,
	tabsize=4,
	showspaces=false,
	showstringspaces=false}


\section*{Aufgabe 18}
Relation "Kunde"\\ \\
\begin{tabular} {|c|c|}
\rowcolor{Gray}\hline
\cellcolor{red!25}\underline{kundenId} \phantom{a}&name\\\hline
1&Aaron Eckhart\\\hline
2&Michael Caine\\\hline
3&Maggie Gyllenhaal\\\hline
\end{tabular}\\

Relation "Pizza"\\ \\
\begin{tabular} {|c|c|c|}
	\rowcolor{Gray}\hline 
	\cellcolor{blue!25}\underline{pizzaId}&name&preis\\\hline
	1&Hollandaise&5,80\\\hline
	2&Cipolla&5,80\\\hline
	3&Tonno&6,80\\\hline
	4&Zingara&6,80\\\hline
	5&Gyros&6,80\\\hline
	6&Toscana&7,50\\\hline
\end{tabular}\\

Relation "Bestellung"\\ \\
\begin{tabular} {|c|c|c|c|}
	\rowcolor{Gray}\hline
	\underline{buchungsId}&buchungsZeitpunkt \phantom{a}&\cellcolor{blue!25}pizzaId&\cellcolor{red!25}kundenId\\\hline
	1&17.06.2017 11:58:29&2&1\\\hline
	2&17.06.2017 14:28:58&5&1\\\hline
	3&17.06.2017 16:19:22&6&2\\\hline
	4&17.06.2017 18:21:31&1&3\\\hline
\end{tabular}\\

Die Fremdschlüssel sind farblich mitsamt Zugehörigkeit zur entsprechenden Relation markiert.
\pagebreak
\lstinputlisting{Kunde.java}
\pagebreak
\lstinputlisting{Pizza.java}
\pagebreak
\lstinputlisting{Bestellung.java}
\pagebreak
\section*{Aufgabe 19}
\includegraphics*[width = \textwidth, height = \textheight, keepaspectratio]{19a.jpeg}
\pagebreak
\lstinputlisting{Gui.java}
\section*{Aufgabe 20}

\begin{enumerate} [a)]
	\item\phantom{}\\ \includegraphics[width=\textwidth, height=\textheight, keepaspectratio]{Aufgabe 20.jpeg}
	\item \phantom{} [rounds, rounds = 0, return rounds] \\
	\phantom{} [rounds, rounds = 0, rounds++] \\
		\phantom{}	[rounds, rounds++, rounds == 2] \\
		\phantom{} [rounds,rounds++,rounds++] \\
\phantom{}	[rounds, rounds++, return rounds ]\\
\phantom{}	[energy, energy = energy - 1, energy > 0] \\
\phantom{}	[energy, energy = energy - 1, energy = energy - 2] \\
\phantom{}	[energy, energy = energy - 1, energy = energy - 1] \\
\phantom{}	[energy, energy = energy - 2, energy > 0] \\
\phantom{}	[energy, energy = energy - 2, energy = energy - 1] \\
\phantom{}	[energy, energy = energy - 2, energy = energy - 2] (nicht erreichbar)\\
\pagebreak
\item  "\phantom{}Energy = 0"\phantom{} durchläuft Kette 1, "\phantom{}Energy = 5"\phantom{} durchläuft Kette 2 - 10.\\  \includegraphics[width=\textwidth, height=\textheight, keepaspectratio]{Aufgabe 20 DU.jpeg}
\end{enumerate}
\end{document}